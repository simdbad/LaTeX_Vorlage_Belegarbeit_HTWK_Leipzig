%%%%%%%%%%%%%%%%%%%%%%%%%%%%%%%%%%
% HTWK Leipzig LaTeX Vorlage - Belegarbeit
% Interne Referenznummer 3_1 Overleaf
%
% Das Dokument basiert teilweise auf den Vorlagen von :
% Markus Voerkel
% https://de.overleaf.com/latex/templates/thesis-template-for-hochschule-fur-technik-wirtschaft-und-kultur-leipzig/nqpftcjmmtts
%
% Linda Vogel & Jon Arnt Kårstad
% https://www.overleaf.com/latex/templates/template-projekt-htwk/mphqwccfvvwy
%
% HTWK Logo:
% https://www.htwk-leipzig.de/hochschule/hochschulkommunikation-marketing/corporate-design
%%%%%%%%%%%%%%%%%%%%%%%%%%%%%%%%%%
%
% Veröffentlicht unter CC BY-SA 4.0 - Lizenz
%
%---------------------------------------
%	Dokumentenklasse
%---------------------------------------

\documentclass[
ngerman,
toc=flat,
toc=chapterentrywithdots,
captions=tableabove,
listof=entryprefix,
listof=leveldown,
fontsize=12pt,
numbers=noenddot,
headsepline]
{scrreprt}

%---------------------------------------
%	Laden von Paketen
%---------------------------------------

\usepackage{babel}
\usepackage{lmodern}
\usepackage[T1]{fontenc}
\usepackage{float}
\usepackage{ragged2e}

% Geoemetry %
\usepackage{geometry}
\geometry{
	top = 27.5mm,
	headsep = 5mm,
	left = 28mm,
	right = 20mm, 
	bottom = 25mm,
	}
	
\usepackage[letterspace=150]{microtype}
\usepackage[onehalfspacing]{setspace}

	
% Caption %
\usepackage[labelfont={bf,sf},font={bf}, labelsep=space, singlelinecheck=off]{caption} 
\captionsetup[figure]{justification=centering}
\captionsetup[table]{justification=raggedright}


\BeforeStartingTOC[toc]{\singlespacing} 
\BeforeStartingTOC[lot]{\renewcommand\autodot{:}}
\BeforeStartingTOC[lof]{\renewcommand\autodot{:}}

% Bibtex %
\usepackage[
    backend=biber,
    style=numeric-comp,
    sorting=none,
    defernumbers=true
]{biblatex}
\addbibresource{Literatur.bib}

\usepackage{csquotes}
\usepackage{amsmath}

\usepackage{graphicx}

% Hyperref %
\usepackage{hyperref}
\hypersetup{ 
    colorlinks=true,
    linkcolor=black,
    urlcolor=blue,
    citecolor=black}

% Einfügen von PDF-Dateien %
\usepackage{pdfpages}

%Einfacher Umgang mit Einheiten%
\usepackage{siunitx}
\sisetup{
    locale=DE,
    per-mode=fraction,
}

% Bessere Kompatibilität der Dokumentenklasse mit div. Paketen %
\usepackage{scrhack}

%---------------------------------------
%	Weitere Konfigurationen
%---------------------------------------


%Anpassung der Kopf- und Fusszeile
\usepackage[autooneside=false]{scrlayer-scrpage}
\clearpairofpagestyles
\setkomafont{pageheadfoot}{\footnotesize}
\ohead{\pagemark}
\automark{chapter}
\ihead{\headmark}

%Anpassung der Kapitelüberschrift
\renewcommand*{\chapterpagestyle}{scrheadings}
\RedeclareSectionCommand[%
beforeskip=0pt,
afterskip=16pt,
afterindent = false,
font=\LARGE]{chapter}

%Verhindere Zeilenumbruch innerhalb \cite[Seite]{quelle}
\renewcommand*{\prenotedelim}{\addnbspace}
\renewcommand*{\postnotedelim}{\addcomma\addnbspace}
\renewcommand*{\multicitedelim}{\addcomma\addnbspace}
\renewcommand*{\extpostnotedelim}{\addnbspace}
\renewcommand*{\volcitedelim}{\addcomma\addnbspace}

%%%%%%%%%%%%%%%%%%%%%%%%%%%%%%%%%%%%%%%%%%%%%%%%%
%
%       Bitte hier die eigenen Daten eingeben
%       für Generierung der Titelseite etc.
%
%%%%%%%%%%%%%%%%%%%%%%%%%%%%%%%%%%%%%%%%%%%%%%%%%

\newcommand{\autor}{Max Mustermann}% Vorname Name
\newcommand{\mnr}{12345} % Matrikelnummer, bspw.: 12345

%%%%%%%%%%%%%%%%%%%%%%%%%%%%%%%%%% 
%
% Hier können weitere Autoren (und eine Gruppe) definiert werden. Diese können durch Anpassung der Titelseite.tex und der Erklaerung.tex berücksichtigt werden:
%
%\newcommand{\autorII}{Erika Musterfrau}% Vorname Name
%\newcommand{\mnrII}{54321} % Matrikelnummer, bspw.: 54321
%
%\newcommand{\gruppenr}{00}
%
%%%%%%%%%%%%%%%%%%%%%%%%%%%%%%%%%%

\newcommand{\betreuerI}{Prof. Dr.-Ing. Moritz Musterprof} % Name betreuender Prof

%\newcommand{\betreuerII}{Dipl.-Ing. Manuela Musteringenieur} % Name 2. Betreuer

\newcommand{\modulname}{Modulname} % Hier den Modulnamen eingeben

\newcommand{\projektname}{Belegtitel} % Hier den Belegtitel eingeben

\newcommand{\datum}{01.\,12.\,2025}

% Fakultäten, nicht ändern!
\newcommand{\FAS}{Architektur und Sozialwissenschaften}
\newcommand{\FB}{Bauwesen}
\newcommand{\FING}{Ingenieurwissenschaften}
\newcommand{\FDIT}{Digitale Transformation}
\newcommand{\FIM}{Informatik und Medien}
\newcommand{\FWW}{Wirtschaftswissenschaft und Wirtschaftsingenieurwesen}

% Auswahl Fakultät:
\newcommand{\fak}{\FING} % Eingabe der Fakultät (siehe oben)

\newcommand{\studiengang}{Maschinenbau} % bspw.: Maschinenbau
 


\input{Konfigurationsdateien/Anhangsverzeichnis}

%Ebenen im Inhaltsverzeichnis
\newcommand{\nocontentsline}[3]{}
\newcommand{\tocless}[2]{\bgroup\let\addcontentsline=\nocontentsline#1{#2}\egroup}

%Verwendung normaler "Gänsefüßchen"
\MakeOuterQuote{"}

%Kein Einzug nach Absatz
\setlength\parindent{0pt}

%Schriftgröße in Tabellen gleich der Schriftgröße im Dokument
\usepackage{etoolbox}
\AtBeginEnvironment{table}{\sffamily}

%%%%%%%%%% Platz für weitere Pakete %%%%%%%%%%
%2 Bilder nebeneinander
\usepackage{subcaption}






%%%%%%%%%%%%%%%%%%%%%%%%%%%%%%%%%%%%%%%%%%%%%

%-------------------------------------------------
%	Beginn des Dokuments und Einbinden der Kapitel
%-------------------------------------------------

\begin{document}

\begin{titlepage}

\begin{center}

\includegraphics[width=0.4\textwidth]{Konfigurationsdateien/HTWK Logo.png}

\vspace{2.5cm}

\textsc{\LARGE Fakultät \fak}
\vspace{1.5cm}

\textsc{\Large \modulname}

\vspace{3cm}

{ \huge \bfseries \projektname}

\vspace{1.5cm}

\renewcommand{\arraystretch}{1.5}
%%%%%%%%%%%%%%%%%%%%%%%%%%%%%%%%%%
%Entsprechend der Anzahl der Autoren anpassen bzw. Kommentarzeichen (%) entfernen

\begin{table}[h]
\large
    \centering
    \begin{tabular}{lll}
        &
        %\underline{Gruppe \gruppenr} %Gruppennummer
        & \\
        \emph{Autor} & \autor & \mnr \\ %ggf. "Autoren"
        %& \autorII & \mnrII \\
        %...
        \vspace{-0.5cm} \\
        \emph{Betreuer} & \betreuerI %& \betreuerII
        
    \end{tabular}
\end{table}
%%%%%%%%%%%%%%%%%%%%%%%%%%%%%%%%%%
\renewcommand{\arraystretch}{1}

\vfill

% Bottom of the page
{\large \datum}
\end{center}
\end{titlepage}
\thispagestyle{empty}
\begin{center}
\large \lsstyle Erklärung
\end{center}
\vspace{1.5cm}
{\doublespacing
Ich versichere wahrheitsgemäß, die Projektarbeit selbständig angefertigt, alle benutzten Hilfsmittel vollständig und genau angegeben und alles kenntlich gemacht zu haben, was aus Arbeiten anderer unverändert oder mit Abänderungen entnommen wurde.}\par
\vspace{2cm}
\noindent
%Autor 1
\begin{minipage}[t]{6.5cm}
% gepunktete Linie
\dotfill

% Text unter der Linie
\onehalfspacing
\autor

Leipzig, den \datum
\end{minipage}\par
\vspace{2.5cm}

%%%%%%%%%%%%%%%%%%%%%%%%%%%%%%%%%%
% Bei mehreren Autoren: 
\iffalse %Block wird von LaTeX ignoriert, für mehrere Autoren \iffalse entfernen
%Autor 2
\begin{minipage}[t]{6.5cm}
\dotfill

\onehalfspacing
\autorII

Leipzig, den \datum
\end{minipage}\par
\vspace{2.5cm}

\fi %Block wird von LaTeX ignoriert, für mehrere Autoren \fi entfernen

%Autor 3
\iffalse % Block wird von LaTeX ignoriert, für mehrere Autoren \iffalse entfernen

\begin{minipage}[t]{6.5cm}
\dotfill

\onehalfspacing
\autorIII

Leipzig, den \datum
\end{minipage}\par

\fi  %Block wird von LaTeX ignoriert, für mehrere Autoren \fi entfernen
%%%%%%%%%%%%%%%%%%%%%%%%%%%%%%%%%%

\clearpage

\tableofcontents
\input{Konfigurationsdateien/Abbildungsverzeichnis}

\input{Hauptkapitel/Kapitel1}
\input{Hauptkapitel/Kapitel2}
\input{Hauptkapitel/Kapitel3}

\begin{singlespacing}
\printbibliography
\end{singlespacing}

\appendix %Entfernen/Auskommentieren um Anhang zu deaktivieren
\input{Anhang} %Entfernen/Auskommentieren um Anhang zu deaktivieren

\end{document}
